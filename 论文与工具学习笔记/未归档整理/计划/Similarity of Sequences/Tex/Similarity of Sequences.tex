% 设置编码,编码为UTF-8编码,字号大小12pt
\documentclass[UTF8, 12pt]{ctexart}
\usepackage{graphicx}
\usepackage{geometry}
\usepackage{titlesec}{\tiny}
\usepackage{amsmath}
\usepackage[colorlinks, linkcolor=blue, citecolor=blue]{hyperref}
% 定义
\newtheorem{theorem}{Theorem}[section]
% 控制图片的位置,让图片紧紧的跟住文字,只需写\begin{figure}[H]
\usepackage{float}
% 使用文献引用
\usepackage{cite}
% 使用算法排版模块
\usepackage{algorithm}  
\usepackage{algorithmic} 

% 设置文本格式,文本间距等,具体参考如下:
% left=2cm, right=2cm, top=2.5cm,bottom=1.5cm
\geometry{a4paper, centering, scale=0.81}
\newtheorem{thm}{定义}


\begin{document}
\title{\heiti 序列相似性度量简介}
\author{\kaishu 尹卓\\北京工业大学\\MichaelYinBJUT@163.com}

\maketitle

% 增加目录
\tableofcontents
\newpage
\section{引言}
\section{轨迹数据示例}

\begin{figure}[H]
	\centering
	\includegraphics[width=0.6\linewidth]{..//Plots//TrajectoryExample.pdf}
	\caption{The original trajectory format}
	\label{trajectoryExample}
	\vspace{-0.5em}
\end{figure}
本文中,使用从西安的交叉口高点监控视频提取的轨迹,演示各个轨迹距离度量的实验效果。其中,轨迹数据的示意图如\ref{trajectoryExample}所示;轨迹数据的格式被定义为:
\begin{thm}[车辆轨迹格式]
	\label{trajectoryFormat}
	一条车辆轨迹数据被表示为按时间先后排序的序列$Traj = \{p_{1}, ..., p_{N}\}$,其中$p_{i} = \{t_{i}, x_{i}, y_{i}, s_{i}, c_{i}\}$。轨迹数据中的$x$与$y$坐标代表车辆在$t_{i}$时刻的位置。$t_{i}$代表标准的$Linux$时间;$s_{i} \in R$代表$t_{i}$时刻车辆的跟踪框的大小;$c_{i} \in \{0, 1\}$代表$t_{i}$时刻,车辆是否位于交叉口内部。对于时间信息而言,$t_{1} < t_{2} < ... < t_{N}$。
\end{thm}

\section{相似性度量}
轨迹数据聚类不同于传统数据的聚类。轨迹数据通常是多维度的,不等长的,时空相关联的数据,这使得传统的聚类算法难以应用于轨迹数据。
\subsection{Euclidean距离}
Euclidean距离通常针对的是等长的轨迹。对于两条不同的轨迹$Traj_{i}$与$Traj_{j}$而言,轨迹的样本点的个数都为$N$个,维数为$P$维。Euclidean距离的计算公式为:
\begin{equation}
	D_{E}(Traj_{i}, Traj_{j}) = \frac{1}{N} \sum_{k=1}^{N} \sqrt{\sum_{m=1}^{P} (a_{k}^{m} - b_{k}^{m})^{2}}
\end{equation}

其中$a_{k}^{m}$代表轨迹$Traj_{i}$的第$k$个点的第$m$个特征的取值。Euclidean距离的时间复杂度为$O(n)$,但是只能用于等长度的轨迹数据,这使得距离度量的应用有一定的困难。

\subsection{DTW距离}


\subsection{LCSS距离}
\subsection{Edit距离}
\subsection{Hausdorff距离}
\subsection{Frechet距离}
\section{总结}

\section{实验}
\end{document}