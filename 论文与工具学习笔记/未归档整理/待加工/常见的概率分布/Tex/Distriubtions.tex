% 设置编码,编码为UTF-8编码,字号大小12pt
\documentclass[UTF8, 12pt]{ctexart}
\usepackage{graphicx}
\usepackage{geometry}
\usepackage{titlesec}{\tiny}
\usepackage{amsmath}
\usepackage{authblk}
% 定义超链接的颜色
\usepackage[colorlinks, linkcolor=blue, citecolor=blue]{hyperref}
% 定义
\newtheorem{theorem}{Theorem}[section]
% 控制图片的位置,让图片紧紧的跟住文字,只需写\begin{figure}[H]
\usepackage{float}
% 使用文献引用
\usepackage{cite}
% 使用算法排版模块
\usepackage{algorithm}  
\usepackage{algorithmic}
\renewcommand{\algorithmicrequire}{\textbf{输入:}}  
\renewcommand{\algorithmicensure}{\textbf{输出:}} 
% 设置文本格式,文本间距等,具体参考如下:
% left=2cm, right=2cm, top=2.5cm,bottom=1.5cm
\geometry{a4paper, centering, scale=0.81}
\newtheorem{thm}{定义}

\begin{document}
\title{\heiti 常见的概率分布的密度函数}
\author{\kaishu 尹卓\\ E-mails: zhuoyin94@163.com}
\date{\today}
\maketitle

% 增加目录
\tableofcontents
\newpage
%%%%%%%%%%%%%%%%%%%%%%%%%%%%%%%%%%%%%%%%%%%%%%%%%%%%%%%%%%%%%%%%%%%%%%%%%%%%%%%%%%%%%%%%%%%%%%%%%%%%%%%%%%%%%%%%%%%%%%%%%%%%%%%%%%%%%%%%%%%%%%%%%%%%%%%%%%%
\section{引言}
本文旨在总结与记录一些LDA中需要运用到的几个概率分布,包括伽马分布,贝塔分布等。

%%%%%%%%%%%%%%%%%%%%%%%%%%%%%%%%%%%%%%%%%%%%%%%%%%%%%%%%%%%%%%%%%%%%%%%%%%%%%%%%%%%%%%%%%%%%%%%%%%%%%%%%%%%%%%%%%%%%%%%%%%%%%%%%%%%%%%%%%%%%%%%%%%%%%%%%%%%
\section{常见分布的分布函数}

\subsection{拉普拉斯分布}
	拉普拉斯分布(Laplace distribution),也被叫做双边指数分布(Double sided exponential),其概率密度函数的形式如下:
	\begin{equation}
		Lap(x|\mu, b) = \frac{1}{2b} exp(-\frac{|x-\mu|}{b})
	\end{equation}
	这里的$\mu$是位置参数,而$b$是尺度参数(Scale parameter)。并且分布具有$mean=\mu$,$variance=b$的性质。注意在Lasso回归当中,参数的先验分布就是拉普拉斯分布。

\subsection{伽马分布}
	伽马分布(Gamma distribution)是一种针对正实数的分布,要求随机变量$x > 0$。决定伽马分布的形状的参数有两个,一个是$a > 0$,另外一个是$b > 0$:
	\begin{equation}
		Ga(x|shape=a, rate=b) = \frac{b^{a}}{\Gamma(a)}x^{a-1}e^{-xb}
	\end{equation}
	其中,$\Gamma(a)$是伽马函数:
	\begin{equation}
		\Gamma(t) = \int_{0}^{\infty} {u^{t-1}e^{-u}}du
	\end{equation}
	伽马分布有这么一些性质:$mean=a/b$,$mode=(a-1)/b$,$var=a/b^{2}$,其中$mode$表示中位数。以下几种分布事实上是伽马分布的特殊形式:
	\begin{enumerate}
		\item 指数分布(Exponential distribution)。指数分布事实上$Expon(x|\lambda) = Ga(x|1, \lambda)$。
		\item 卡方分布(Chi-square distribution)。卡方分布事实上等价于$\chi^{2}(x|\nu) = Ga(x|\nu/2, 1/2)$。而卡方分布定义为:$S=\sum_{i=1}^{\nu}{Z_{i}^{2}}$,则$S \sim \chi_{\nu}^{2}$,其中$Z_{i} \sim N(0, 1)$。
	\end{enumerate}
	还有一个有用的结论是:若是随机变量$X \sim Ga(a, b)$,则可得$1/X \sim IG(a, b)$,其中IG是逆伽马分布(Inverse gamma distribution)。定义为:
	\begin{equation}
		IG(x| shape=a, scale=b) = \frac{b^{a}}{\Gamma(a)}x^{-(a+1)}e^{-b/x}
	\end{equation}
	并且:$mean=b/(a-1)$,$mode=b/(a+1)$,$var=b^{2}/(a-1)^{2}(a-2)$。其中均值存在的条件是$a > 1$,方差存在的条件是$a > 2$。
	
	介绍了伽马分布,着重介绍一下伽马函数:
	\begin{enumerate}
		\item 伽马函数的递推公式可以写为:$\Gamma(x + 1) = x \Gamma(x)$。并且对于正整数$n$,一定存在:
		\begin{equation}
			\Gamma(n + 1) = n!
		\end{equation}
		递推公式的推导为:
		\begin{align}
			\Gamma(n + 1) = & \int_{0}^{\infty}{ e^{-x} x^{n + 1 - 1} }dx = \int_{0}^{\infty} {e^{-x} x^{n}}dx \\
						  = & \int_{0}^{\infty} {e^{-x} x^{n}}dx = [\frac{-x^{n}}{e^{x}}]_{0}^{\infty} + n\int_{0}^{\infty} {e^{-x} x^{n-1}}dx
		\end{align}
		关键的一项推导是:当$x = 0$的时候,$-0^{n}/e^{0} = 0$。当$x \rightarrow \infty$时,根据洛必达法则:
		\begin{equation}
			\lim\limits_{x \rightarrow \infty} \frac{-x^{n}}{e^{x}} = \lim\limits_{x \rightarrow \infty} \frac{-n!}{e^{x}} = 0
		\end{equation}
		由此获得递推公式。
		\item 乘法定理:
		\begin{align}
			& \Gamma(x)\Gamma(x + 1/2) = 2^{1 - 2z} \sqrt{\pi} \Gamma(2z) \\
			& \Gamma(x)\Gamma(x + \frac{1}{m})...\Gamma(x + \frac{m - 1}{m}) = (2\pi)^{(m-1)/2} m^{1/2 - mx} \Gamma(mx)
		\end{align}
	\end{enumerate}

\subsection{贝塔分布}
	贝塔分布(Beta distribution)被定义为:
	\begin{equation}
		Beta(x|a, b) = \frac{1}{B(a, b)} x^{a-1}(1-x)^{b-1}
	\end{equation}
	其中$B(a, b)$是贝塔函数:
	\begin{equation}
		B(a, b) = \frac{\Gamma(a)\Gamma(b)}{\Gamma(a + b)}
	\end{equation}
	其中要求$a, b > 0$。若是$a = b = 1$,我们可以得到均匀分布(Uniform distribution);若是$a$与$b$都小于1, 我们可以得到峰值在0与1的双峰分布(Bimodal distribution);若是$a$与$b$都大于1,则分布是单峰分布(Unimodal distribution)。分布的性质有:$mean = a/(a + b)$,$mode = (a - 1)/(a + b - 2)$,$variance = ab/(a + b)^2(a + b + 1)$。
	
%\bibliographystyle{unsrt}  
%\bibliography{DBSCAN_ref}  
\end{document}