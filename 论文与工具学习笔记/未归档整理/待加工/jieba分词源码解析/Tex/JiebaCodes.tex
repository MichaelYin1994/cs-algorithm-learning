% 设置编码,编码为UTF-8编码,字号大小12pt
\documentclass[UTF8, 12pt]{ctexart}
\usepackage{graphicx}
\usepackage{geometry}
\usepackage{titlesec}{\tiny}
\usepackage{amsmath}
\usepackage{authblk}
% 定义超链接的颜色
\usepackage[colorlinks, linkcolor=blue, citecolor=blue]{hyperref}

% 标题左对齐
%\CTEXsetup[format={\Large\bfseries}]{section}

% 定义
\newtheorem{theorem}{Theorem}[section]
% 控制图片的位置,让图片紧紧的跟住文字,只需写\begin{figure}[H]
\usepackage{float}
% 使用文献引用
\usepackage{cite}
% 使用算法排版模块
\usepackage{algorithm}  
\usepackage{algorithmic}
\renewcommand{\algorithmicrequire}{\textbf{输入:}}  
\renewcommand{\algorithmicensure}{\textbf{输出:}} 
% 设置文本格式,文本间距等,具体参考如下:
% left=2cm, right=2cm, top=2.5cm,bottom=1.5cm
\geometry{a4paper, centering, scale=0.8}
\newtheorem{thm}{定义}
\renewcommand{\baselinestretch}{1.3}

% 定义编程语言
\usepackage{listings}
\usepackage{color}
\usepackage{fontspec}
\definecolor{dkgreen}{rgb}{0,0.6,0}
\definecolor{gray}{rgb}{0.5,0.5,0.5}
\definecolor{mauve}{rgb}{0.58,0,0.82}
\definecolor{light-gray}{gray}{0.95}

\lstset{frame=tb,
		language=Python,
		aboveskip=3mm,
		belowskip=1mm,
		showstringspaces=false,
		columns=flexible,
		basicstyle=\small\ttfamily,
		numbers=left,
		numberstyle=\small\color{gray},
		keywordstyle=\color{blue},
		commentstyle=\color{dkgreen},
		stringstyle=\color{mauve},
		breaklines=true,
		breakatwhitespace=true,
		tabsize=4,
		backgroundcolor=\color{light-gray}
}

\begin{document}

	
	\title{\heiti \Huge{Jieba分词原理浅析}}
	\author{\kaishu 尹卓 \\ E-mails \href{mailto:zhuoyin94@163.com}{zhuoyin94@163.com}}
	\date{\today}
	\maketitle
	
	% 增加目录
	\tableofcontents
	\newpage
	\section{引言}
	
	
	\begin{enumerate}
		\item “沙瑞金赞叹易学习的胸怀,是金山的百姓有福,可是这件事对李达康的触动很大。”
		\item “易学习又回忆起他们三人分开的前一晚,大家一起喝酒话别,易学习被降职到道口县当县长,王大路下海经商,李达康连连赔礼道歉,觉得对不起大家,他最对不起的是王大路,就和易学习一起给王大路凑了5万块钱,王大路自己东挪西撮了5万块,开始下海经商。没想到后来王大路竟然做得风生水起。”
		\item “沙瑞金向毛娅打听他们家在京州的别墅,毛娅笑着说,王大路事业有成之后,要给欧阳菁和她公司的股权,她们没有要,王大路就在京州帝豪园买了三套别墅,可是李达康和易学习都不要,这些房子都在王大路的名下,欧阳菁好像去住过,毛娅不想去,她觉得房子太大很浪费,自己家住得就很踏实。”
	\end{enumerate}
	
	\section{结巴分词的功能与接口}
	jieba.lcut
	
	\section{结巴分词的程序流程}
	结巴分词的流程:正则粗分,载入用户词典,DAG细分
	
	\section{Trie树到前向哈希}
	github pull request索引词频,构建Hash表。(源代码)
	
	\section{DAG的最大概率切分}
	源代码解析
	
	\section{基于HMM的新词发现}
	源代码解析
	
	\section{总结}
	Jieba分词的优缺点。
	
	
\end{document}