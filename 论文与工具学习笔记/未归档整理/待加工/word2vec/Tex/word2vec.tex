% 设置编码,编码为UTF-8编码,字号大小12pt
\documentclass[UTF8, 12pt]{ctexart}
\usepackage{graphicx}
\usepackage{geometry}
\usepackage{titlesec}{\tiny}
\usepackage{amsmath}
\usepackage{authblk}
\usepackage{amsfonts}
% 定义超链接的颜色
\usepackage[colorlinks, linkcolor=blue, citecolor=blue]{hyperref}

% 标题左对齐
%\CTEXsetup[format={\Large\bfseries}]{section}

% 定义
\newtheorem{theorem}{Theorem}[section]
% 控制图片的位置,让图片紧紧的跟住文字,只需写\begin{figure}[H]
\usepackage{float}
% 使用文献引用
\usepackage{cite}
% 使用算法排版模块
\usepackage{algorithm}  
\usepackage{algorithmic}
\renewcommand{\algorithmicrequire}{\textbf{输入:}}  
\renewcommand{\algorithmicensure}{\textbf{输出:}} 
% 设置文本格式,文本间距等,具体参考如下:
% left=2cm, right=2cm, top=2.5cm,bottom=1.5cm
\geometry{a4paper, centering, scale=0.8}
\newtheorem{thm}{定义}
\renewcommand{\baselinestretch}{1.3}

% 定义编程语言
\usepackage{listings}
\usepackage{color}
\usepackage{fontspec}
\definecolor{dkgreen}{rgb}{0,0.6,0}
\definecolor{gray}{rgb}{0.5,0.5,0.5}
\definecolor{mauve}{rgb}{0.58,0,0.82}
\definecolor{light-gray}{gray}{0.95}

\lstset{frame=tb,
		language=Python,
		aboveskip=3mm,
		belowskip=1mm,
		showstringspaces=false,
		columns=flexible,
		basicstyle=\small\ttfamily,
		numbers=left,
		numberstyle=\small\color{gray},
		keywordstyle=\color{blue},
		commentstyle=\color{dkgreen},
		stringstyle=\color{mauve},
		breaklines=true,
		breakatwhitespace=true,
		tabsize=4,
		backgroundcolor=\color{light-gray}
}

\begin{document}
	
\title{\heiti \Huge{word2vec简介}}
\author{\kaishu 尹卓 \\ \href{mailto:zhuoyin94@163.com}{zhuoyin94@163.com}}
\date{\today}
\maketitle

% 增加目录
\tableofcontents
\newpage
\section{引言}
	这里是引言。
	
\section{连续词袋模型(CBoW)}
	
	\subsection{单词的连续词袋模型}
	单词的连续词袋模型只考虑一个词,也就意味着给定一个\emph{中心词(Context word)},只预测一个\emph{目标词(Target word)}。假设我们的词表的大小为$V$,隐层神经元的个数为$N$,该神经网络为一个\emph{全连接(Fully Connected)}的神经网络。神经网络的输入是经过独热码的词向量,意即对于输入$\textbf{x} = \{x_{i}\}_{i=1}^{V}$而言,只有中心词位置为1,其他都是0;对于需要预测的输出$\textbf{y}$而言,只有目标词位置为1,其他都是0。
	
	输入层到隐藏层的权重定义为$\textbf{W} \in \mathbb{R}^{V \times N}$。$\textbf{W}$矩阵的每一行被定义为$\textbf{v}_{w}^{T}$,对于输入层到隐层的运算而言,假设对于向量$\textbf{x}$而言,$x_{k}=1$,其他位置都为0,便有:
	\begin{equation}
		\label{hiddenUnitVec}
		\textbf{h} = \textbf{W}^{T}\textbf{x}=\textbf{v}_{w_{I}}
	\end{equation}
	这样意味着,隐层向量$\textbf{h} \in \mathbb{R}^{N \times 1}$是$\textbf{W}$矩阵的第$k$行的一份拷贝。
	
	对于隐层到输出层的而言,设其权重为$\textbf{W}^{'} \in \mathbb{R}^{N \times V}$。对于输出层的第$j$个输出而言,其表达式为:
	\begin{equation}
		\label{outputUnit_j}
		u_{j} = {\textbf{v}^{'}_{w_{j}}}^{T} \textbf{h}
	\end{equation}
	其中,$\textbf{v}^{'}_{w_{j}}$为矩阵$\textbf{W}^{'}$的第$j$列。在有了以上的表达式之后,我们使用softmax模型,获得输出词的后验概率:
	\begin{equation}
		\label{posteriorDist}
		p(w_{j}|w_{I}) = y_{j} = \frac{\exp(u_{j})}{\sum_{j^{'} = 1}^{V} \exp(u_{j^{'}})}
	\end{equation}
	其中$y_{j}$是神经网络的第$j$个输出。在获得了式(\ref{hiddenUnitVec})与(\ref{outputUnit_j})的表达式之后,可以将二式代入到(\ref{posteriorDist})中去。便可以得到:
	\begin{equation}
		p(w_{j}|w_{I}) = y_{j} = \frac{\exp({\textbf{v}^{'}_{w_{j}}}^{T} \textbf{v}_{w_{I}})}{\sum_{j^{'} = 1}^{V} \exp({\textbf{v}^{'}_{w_{j^{'}}}}^{T} \textbf{v}_{w_{I}})}
	\end{equation}
	在这里应当注意到,向量$\textbf{v}_{w}$与向量$\textbf{v}^{'}_{w}$是词的两种不同的表达形式。向量$\textbf{v}_{w}$是矩阵$\textbf{W}^{T}$的一列,向量$\textbf{v}^{'}_{w}$是矩阵$\textbf{W}^{'}$的一列。我们一般将$\textbf{v}_{w}$称为\emph{输入向量(Input vector)},向量$\textbf{v}^{'}_{w}$称为\emph{输出向量(Output vector)}。
	
	我们此处考虑交叉熵损失函数:
	\begin{equation}
		\label{crossEntropyLoss}
		L(\textbf{y}, \hat{\textbf{y}}) = -\sum_{j^{'} = 1}^{V} y_{j^{'}}\log(\hat{y}_{j^{'}})
	\end{equation}
	式(\ref{crossEntropyLoss})中的$y_{j^{'}}$是第$j^{'}$个神经元的真实输出,而$\hat{y}_{j^{'}}$是神经网络的网络输出。由于目标词只有一个,所以该损失函数可以被进一步简化为最小化式(\ref{posteriorDist})的后验概率,意即:
	\begin{align}
		\min - {p(w_{j}|w_{I})} & = \min - y_{j^{*}} \\
							    & = \min - \log y_{j^{*}} \\
							    & = \log \sum_{j^{'} = 1}^{V} \exp(u_{j^{'}}) - u_{j^{*}} \\
							    & = E
	\end{align}
	其中$j^{*}$代表实际有输出值的神经元。
	
	接下来,考虑由输出层$\Rightarrow$隐藏层的反向传播过程。损失$E$对于第$j$个神经元的偏导数可以写为:
	\begin{equation}
		\frac{\partial E}{\partial u_{j}} = y_{j} - t_{j} = e_{j}
	\end{equation}
	接下来对$\textbf{W}^{'}$矩阵的元素$w_{ij}^{'}$进行求导,获取权值的更新公式:
	\begin{equation}
		\frac{\partial E}{\partial w_{ij}^{'}} = \frac{\partial E}{\partial u_{j}} \cdot \frac{\partial u_{j}}{\partial w_{ij}^{'}} = e_{j} \cdot h_{i}
	\end{equation}
	其中:
	\begin{equation}
		u_{j} = \sum_{i=1}^{N} w_{ij}^{'} h_{i}
	\end{equation}
	因此,$w_{ij}^{'}$的梯度更新公式可以写为:
	\begin{equation}
		{w_{ij}^{'}}^{(new)} = {w_{ij}^{'}}^{(old)} - \eta \cdot e_{j} \cdot h_{i}
	\end{equation}
	或者是向量形式:
	\begin{equation}
		{\textbf{v}_{w_{j}}^{'}}^{(new)} = {\textbf{v}_{w_{j}}^{'}}^{(old)} - \eta \cdot e_{j} \cdot \textbf{h}
	\end{equation}
	其中$j=1,2,...,V$,$\eta$代表学习率。
	
	接下来,考虑由隐藏层$\Rightarrow$输入层的反向传播过程。考虑$E$对于隐层单元的偏导数可以写为:
	\begin{equation}
		\frac{\partial E}{\partial h_{i}} = \sum_{j=1}^{V} \frac{\partial E}{\partial w_{ij}^{'}} \cdot \frac{\partial u_{j}}{\partial h_{i}} =
		\sum_{j=1}^{V} e_{j} \cdot w_{ij}^{'} = EH_{i}
	\end{equation}
	其中$e_{j} = y_{j} - t_{j}$是第$j$个神经元的预测误差。而$EH_{i}$组成的向量$\textbf{EH} \in \mathbb{R}^{N \times 1}$是一个$N$维的向量。并且我们有:
	\begin{equation}
		h_{i} = \sum_{k=1}^{N} x_{k} w_{ki}
	\end{equation}
	结合上面二式可以获得$w_{ki}$的更新公式:
	\begin{equation}
		\frac{\partial E}{\partial w_{ki}} = \frac{\partial E}{\partial h_{i}} \cdot \frac{\partial h_{i}}{\partial w_{ki}} = EH_{i} \cdot x_{k}
	\end{equation}
	这事实上等价于:
	\begin{equation}
		\frac{\partial E}{\partial \textbf{W}} = \textbf{x}\textbf{EH}^{T}
	\end{equation}
	
	% word2vec程序实现
	% https://towardsdatascience.com/an-implementation-guide-to-word2vec-using-numpy-and-google-sheets-13445eebd281
	
	\subsection{多词的连续词袋模型}
	
\section{跳字模型(Skip-Gram)}
	
	\subsection{单词的跳字模型}
	
	\subsection{多词的跳字模型}

\section{word2vec的Python实现:简单代码}
	
\section{word2vec的Python实现:gensim}


\section{总结}

\bibliographystyle{plain}  
\bibliography{word2vec_ref}  
\end{document}