%!TEX program = xelatex
% 完整编译: xelatex -> bibtex -> xelatex -> xelatex
\documentclass[lang=cn,11pt,a4paper,cite=number]{elegantpaper}

\title{标题第一行 \\ 标题第二行}
\author{尹卓 \\ \email{zhuoyin94@163.com}} 

\date{\zhtoday}

% 本文档命令
\usepackage{array}
\newcommand{\ccr}[1]{\makecell{{\color{#1}\rule{1cm}{1cm}}}}

\begin{document}

%%%%%%%%%%%%%%%%%%%%%%%%%%
%%%%%%%%%%%%%%%%%%%%%%%%%%
\maketitle
\begin{abstract}

这里是摘要。
\keywords{关键词1、关键词2}
\end{abstract}

%%%%%%%%%%%%%%%%%%%%%%%%%%
%%%%%%%%%%%%%%%%%%%%%%%%%%
\section{模板使用须知}

% ***********
\subsection{模板介绍}

此模板基于 \LaTeX{} 的标准文类 article 设计,所以 article 文类的选项也能传递给本模板,比如 \lstinline{a4paper, 11pt} 等等。本模板支持 \hologo{pdfLaTeX} 和 \hologo{XeLaTeX} 编译。

\begin{lstlisting}
\documentclass[a4paper,11pt]{elegantpaper}
\end{lstlisting}

\textbf{注意}:Elegant\LaTeX{} 系列模板已经全部上传至 \href{https://www.overleaf.com/latex/templates/elegantpaper-template/yzghrqjhmmmr}{Overleaf} 上,用户可以在线使用。另外,为了方便国内用户,模板也已经传至\href{https://gitee.com/ElegantLaTeX/ElegantPaper}{码云}。

% ***********
\subsection{数学字体选项}

本模板定义了一个数学字体选项(\lstinline{math}),可选项有三个:
\begin{enumerate}
  \item \lstinline{math=cm}(默认),使用 \LaTeX{} 默认数学字体(推荐,无需声明);
  \item \lstinline{math=newtx},使用 \lstinline{newtxmath} 设置数学字体(潜在问题比较多)。
  \item \lstinline{math=mtpro2},使用 \lstinline{mtpro2} 宏包设置数学字体,要求用户已经成功安装此宏包。
\end{enumerate}

% ***********
\subsection{中文字体选项}
模板提供中文字体选项 \lstinline{chinesefont},可选项有
\begin{enumerate}
\item \lstinline{ctexfont}:默认选项,使用 \lstinline{ctex} 宏包根据系统自行选择字体,可能存在字体缺失的问题,更多内容参考 \lstinline{ctex} 宏包\href{https://ctan.org/pkg/ctex}{官方文档}\footnote{可以使用命令提示符,输入 \lstinline{texdoc ctex} 调出本地 \lstinline{ctex} 宏包文档}。
\item \lstinline{founder}:方正字体选项,调用 \lstinline{ctex} 宏包并且使用 \lstinline{fontset=none} 选项,然后设置字体为方正四款免费字体,方正字体下载注意事项见后文。
\item \lstinline{nofont}:调用 \lstinline{ctex} 宏包并且使用 \lstinline{fontset=none} 选项,不设定中文字体,用户可以自行设置中文字体,具体见后文。
\end{enumerate}

\noindent \textbf{注意:} 使用 \lstinline{founder} 选项或者 \lstinline{nofont} 时,必须使用 \hologo{XeLaTeX} 进行编译。

% ***********
\subsection{自定义命令}
此模板并没有修改任何默认的 \LaTeX{} 命令或者环境\footnote{目的是保证代码的可复用性,请用户关注内容,不要太在意格式,这才是本工作论文模板的意义。}。另外,我自定义了 4 个命令:
\begin{enumerate}
  \item \lstinline{\email}:创建邮箱地址的链接,比如 \email{ddswhu@outlook.com};
  \item \lstinline{\figref}:用法和 \lstinline{\ref} 类似,但是会在插图的标题前添加 <\textbf{图 n}> ;
  \item \lstinline{\tabref}:用法和 \lstinline{\ref} 类似,但是会在表格的标题前添加 <\textbf{表 n}>;
  \item \lstinline{\keywords}:为摘要环境添加关键词。
\end{enumerate}

% ***********
\subsection{参考文献}
此模板使用 \hologo{BibTeX} 来生成参考文献,中文模式下默认使用的文献样式(bib style)是 \lstinline{GB/T 7714-2015}\footnote{通过调用 \href{https://ctan.org/pkg/gbt7714}{\lstinline{gbt7714}} 宏包}。参考文献示例:\cite{en3} 使用了中国一个大型的 P2P 平台(人人贷)的数据来检验男性投资者和女性投资者在投资表现上是否有显著差异。

本模板还添加了 \lstinline{cite=numbers} 、\lstinline{cite=super} 和 \lstinline{cite=authoryear}  三个参考文献选项,用于设置参考文献格式的设置,默认为 \lstinline{numbers}。理工科类一般使用数字形式 \lstinline{numbers} 或者上标形式 \lstinline{super},而文科类多使用作者-年份 \lstinline{authoryear} 比较多。如果需要改为 \lstinline{cite=numbers}  或者  \lstinline{authoryear} ,可以使用
\begin{lstlisting}
\documentclass[cite=super]{elegantpaper} % super style ref style
\documentclass[super]{elegantpaper}

\documentclass[cite=authoryear]{elegantpaper} % author-year ref style
\documentclass[authoryear]{elegantpaper}
\end{lstlisting}

\begin{table}[!htb]
  \centering
  \caption{Elegant\LaTeX{} 系列模板捐赠榜}
    \begin{tabular}{*{4}{>{\scriptsize}c}|*{4}{>{\scriptsize}c}}
    \hline
    \textbf{捐赠者} & \textbf{金额} & \textbf{时间} & \textbf{渠道} & \textbf{捐赠者} & \textbf{金额} & \textbf{时间} & \textbf{渠道} \\
    \hline
    Lerh  & 10 RMB & 2019/05/15 & 微信    & 越过地平线 & 10 RMB & 2019/05/15 & 微信 \\
    \hline
    \end{tabular}%
  \label{tab:donation}%
\end{table}%

%%%%%%%%%%%%%%%%%%%%%%%%%%
%%%%%%%%%%%%%%%%%%%%%%%%%%
\nocite{*}
\bibliography{wpref}

\appendix
%\appendixpage
\addappheadtotoc
\section{使用 newtx 系列字体}

如果需要使用原先版本的 \lstinline{newtx} 系列字体,可以通过显示声明数学字体:

\begin{lstlisting}
\documentclass[math=newtx]{elegantbook}
\end{lstlisting}

\end{document}
